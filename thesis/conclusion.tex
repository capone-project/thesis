\chapter{Conclusion}

In this thesis, we have developed an approach tackling the problem of connecting distributed resources with each other.
These resources are presented by services, where each service exposes specialized functionality to possible clients.
All participants of the protocol have an identity, represented by a public key, which is used to authenticate and secure communication of both client and server.

The protocol makes available all functionality by establishing sessions with services.
Each session is thereby protected by identity-bound capabilities, which can be securely distributed to other clients or services, which makes the protocol steerable by a controlling application.
These capabilities can also be requested for identities other than the application's identity.
This mechanism allows an application to create a new session and then pass on the capability to another client.
Having received the capability, this client is now able to connect to the service and use its functionality.

The current implementation provides five different services with targeted functionality.
Besides more common services like sharing a display, forwarding input or executing remote commands, we have also developed two services specific to our protocol.
While the Invoke service handles the use-cases of pushing a capability to another service, which shall then execute the service for which the capability is designated for, the Capability service handles the case where a capability is to be requested from another client having control over a service.

Benchmarks performed have shown that the protocol is efficient enough to handle both interactive work-flows with acceptable latency as well as scenarios where higher throughput is required.

The complete project is publicly available under the copyleft GNU General Public License Version 3.
It is currently available for download at \url{https://github.com/pks-t/sd}.

% vim: ft=tex tw=0
