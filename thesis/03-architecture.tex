\section{Architecture}

\tikzset{
    host/.style={
        draw,
        rectangle
    },
    switch/.style={
        draw,
        rectangle
    },
    hwslave/.style={
        draw,
        rectangle
    },
    appslave/.style={
        draw,
        ellipse
    },
    netlink/.style={
        draw,
        line width=0.4mm,
        triangle 45-triangle 45
    }
}

\begin{figure}[H]
    \centering

    \begin{tikzpicture}
        \node[host] (host) {HOST};
        \node[switch,above=of host] (switch) {SWITCH};

        \node[hwslave,above=4cm of switch] (network) {NETWORK};
        \node[draw,cloud,above=1cm of network] (inet) {INET};

        \node[hwslave,left=2cm of switch,yshift=+1.0cm] (mouse)    {MOUSE};
        \node[hwslave,left=2cm of switch,yshift=+2.0cm] (keyboard) {KEYBOARD};
        \node[hwslave,left=2cm of switch,yshift=+3.0cm] (storage)  {STORAGE};
        \node[hwslave,left=2cm of switch,yshift=+4.0cm] (gpu)      {GPU};

        \node[appslave,right=2cm of switch,yshift=+1.0cm] (app1) {APP1};
        \node[appslave,right=2cm of switch,yshift=+2.0cm] (app2) {APP2};
        \node[right=2cm of switch,yshift=+3.0cm]          (appi) {$\ldots$};
        \node[appslave,right=2cm of switch,yshift=+4.0cm] (appn) {APP$n$};

        \path[netlink] (host) to (switch.south);

        \path[netlink] (switch) to (network);
        \path[netlink] (network) to (inet);

        \path[netlink] ([xshift=0.00cm]switch.north west) |- (mouse.east);
        \path[netlink] ([xshift=0.25cm]switch.north west) |- (keyboard.east);
        \path[netlink] ([xshift=0.50cm]switch.north west) |- (storage.east);
        \path[netlink] ([xshift=0.75cm]switch.north west) |- (gpu.east);

        \path[netlink]        ([xshift=-0.00cm]switch.north east) |- (app1.west);
        \path[netlink]        ([xshift=-0.25cm]switch.north east) |- (app2.west);
        \path[netlink,dashed] ([xshift=-0.50cm]switch.north east) |- (appi.west);
        \path[netlink]        ([xshift=-0.75cm]switch.north east) |- (appn.west);
    \end{tikzpicture}

    \caption{Architecture}
\end{figure}

\subsection{Connection flowchart}

\begin{figure}[H]
    \centering

    \begin{tikzpicture}
        [
            line/.style = {
                draw,
                -latex'
            },
            block/.style = {
                rectangle,
                draw,
                fill=blue!20,
                text width=5em,
                text centered,
                rounded corners,
                minimum height=4em
            },
            decision/.style = {
                diamond,
                draw,
                fill=blue!20,
                text width=4.5em,
                text badly centered,
                inner sep=0pt
            }
        ]

        \node[block] (attach) {App attaches};
        \node[block,below=of attach] (announce) {App announces};
        \node[decision,below=of announce] (satisfiable) {Reqs satisfiable?};
        \node[block,right=of satisfiable] (unsatisfied) {Fail due to unsatisfiability};
        \node[block,below=of satisfiable] (askpeers) {Notify peer devices};
        \node[decision,below=of askpeers] (ack) {All devs acknowledge?};
        \node[block,right=of ack] (nack) {Fail due to nack};
        \node[block,below=of ack] (start) {Notify app of start};
        \node[block,below=of start] (negotiate) {Negotiate secure connections with peers};
        \node[block,below=of negotiate] (run) {Run};

        \path[line] (attach) -- (announce);
        \path[line] (announce) -- (satisfiable);
        \path[line] (satisfiable) -- node[near start,below] {no?} (unsatisfied);
        \path[line] (satisfiable) -- node[near start,right] {yes?} (askpeers);
        \path[line] (askpeers) -- (ack);
        \path[line] (ack) -- node[near start,below] {no?} (nack);
        \path[line] (ack) -- node[near start,right] {yes?} (start);
        \path[line] (start) -- (negotiate);
        \path[line] (negotiate) -- (run);
    \end{tikzpicture}

    \caption{Connecting an app}
\end{figure}

\subsection{Low-level protocol}
\label{sec:low-level-protocol}

The low-level protocol determines how the actual packets sent between client and server are constructed.
It specifies a common format independent of the actual transport layer that is used to guarantee a common understanding of how to handle incoming bytes.
The protocol is currently designed to be used over either TCP or UDP but should be generic enough to be used for other transport layers.
We will concentrate on describing design choices based on TCP and UDP transport layours only, though.

\subsubsection{Package boundaries}

The major difference between TCP and UDP is that TCP is stream based and UDP is packet based.
This means that UDP packages have their size attached to the package itself, so it is guaranteed that when receiving a single package over UDP we know package boundaries and thus can split incoming packages by these boundaries.
On TCP this is not true, though, so we need to design a package format which is able to specify package boundaries so the client is able to determine a unit.

The initial design simply prefixed every single package with an unencrypted package length fixed to four bytes in network byte order.
This allowed the client to initially receive four bytes, convert these to host byte order and subsequently receive the amount of bytes specified.
While easy to implement this has several disadvantages.

The most obvious disadvantage is that the package length was always transmitted unencrypted and without any message authentication code.
This allows potential adversaries to do easily traffic analysis based on package lengths or drive an attack by simply tampering with the length.
As the receiver of the package cannot verify the package's length in any way he has to trust it and thus may receive invalid packages.

In order to solve the problem the most obvious solution would be to simply encrypt the length and attach a message authentication code.
By doing so we would prevent the adversary of knowing the following data's length and prevent that he is able to tamper with it.
Like this, we obviously gain a boost in security compared to sending package length's in plain text.

On the other hand, the adversary is still able to observe actual package lengths.
In the case of UDP being used as a transport layer this is trivial:
An adversary can simply inspect package headers and thus extract the actual package length, as data is in the general case sent in a single package.
In the case of TCP where no actual package length is encoded in the TCP header this is a little bit more involved but in many cases still feasible as client and server are likely to exchange single messages by turn.
So by substracting the prefixed package length and its message authentication code from the total bytes sent in one turn we are able to gain knowledge of the actual message's length with a high probability.

Due to this we are effectively giving the adversary an decryption oracle as he is able to correlate package lengths and their encrypted representations.
Furthermore it becomes much easier for an adversary to analyze traffic by inspecting actual package lengths.
\\\\

To fix the problem we instead send blocks of a fixed length instead of prefixing every package with an encrypted length.
The initial fixed-length block is prefixed with the overall length of the assembled package.
If the overall length exceeds the length of the remaining bytes of the first fixed length block, then the complete block is assembled by concatenating the first block excluding its length prefix and all subsequent blocks until the announced length is received.
The last block is right-padded with zeroes until its length matches the block length.
The package length is only encoded in the first block.

The following example demonstrates the package format for unencrypted packages.
We assume a fixed length block size of 64 bytes.
If a party now wishes to transfer a package of 80 bytes the package will get split into two blocks.
The first block is prefixed with four bytes containing the actual package length and filled with the initial $64 - 4 = 60$ bytes.
The second package is filled with the remaining $80 - 60 = 20$ bytes and right-padded with $64 - 20 = 44$ zeroes.
Figure \ref{fig:unencrypted-package-format} visualizes the block's contents.

\begin{figure}
    \center

    \begin{tikzpicture}[
            every node/.style={ minimum width=6mm, minimum height=8mm },
            start chain=1 going right,
            start chain=2 going right,
            node distance=-0.15mm
        ]

        \node[draw, on chain=1, minimum width=8mm] {len};
        \node[draw, on chain=1, minimum width=88mm] {60 bytes of data};
        \node[left=1cm of 1-1] {Package 1};

        \node[draw, on chain=2, minimum width=40mm, below=1cm of 1-1.west, anchor = west] {20 bytes of data};
        \node[draw, on chain=2, minimum width=56mm] {padding};
        \node[left=1cm of 2-1] {Package 2};
    \end{tikzpicture}

    \caption{Unencrypted package format}
    \label{fig:unencrypted-package-format}
\end{figure}

The second example (see figure \ref{fig:encrypted-package-format}) demonstrates the package format for an encrypted connection.
In contrast to unencrypted packages there is an additional message authentication code attached to each split package which is now assumed to be 16 bytes long.

\begin{figure}
    \center

    \begin{tikzpicture}[
            every node/.style={ minimum width=6mm, minimum height=8mm },
            start chain=1 going right,
            start chain=2 going right,
            node distance=-0.15mm
        ]

        \node[draw, on chain=1, minimum width=24mm] {MAC};
        \node[draw, on chain=1, minimum width=8mm ] {len};
        \node[draw, on chain=1, minimum width=64mm] {44 bytes data};
        \node[left=1cm of 1-1] {Package 1};

        \node[draw, on chain=2, minimum width=24mm, below=1cm of 1-1.west, anchor = west] {MAC};
        \node[draw, on chain=2, minimum width=54mm] {36 bytes data};
        \node[draw, on chain=2, minimum width=18mm] {padding};
        \node[left=1cm of 2-1] {Package 2};

        \draw [decorate,decoration={brace,amplitude=10pt}] (1-2.north west) -- (1-3.north east) node[midway,yshift=7mm] {encrypted};
        \draw [decorate,decoration={brace,amplitude=10pt,mirror}] (2-2.south west) -- (2-3.south east) node[midway,yshift=-7mm] {encrypted};
    \end{tikzpicture}

    \caption{Encrypted package format}
    \label{fig:encrypted-package-format}
\end{figure}

The client will now fetch the initial fixed-size block
If the connection is encrypted, he will decrypt it and verify its message authentication code.
Now we inspect the first four bytes to gain knowledge about the total package length -- if it exceeds the amount of bytes sent in the initial message, we will receive, decrypt and verify all following packages until all bytes are received.
Finally, we concatenate the blocks to obtain the complete package.

\subsubsection{Cryptography}

All connections except the initial service discovery are authenitcated and encrypted to keep information safe.
Encryption is done via secret-key authenticated encryption with an encryption key shared between both parties.
The key is an ephemeral key generated in an authenticated way via Diffie-Hellman key exchange when establishing the connection (see section \ref{sec:connection-establishment} for more details on the actual key exchange).

As the generated key is an ephemeral key which is not to be re-used in later sessions we are able to use a simple counter mode for nonces.
That is, on initial connection the client per definition has a nonce of value $0$ and the server has a nonce of value $1$.
Whenever either the client or server encrypts and sends a new block, its nonce is incremented by $2$.
On the receiving side we increment the remote nonce by $2$ every time we receive an encrypted block.
This guarantees (given nonces of sufficient length) that nonces are never repeated as collisions between client and server nonce are impossible and nonces are monotonously increasing.

The algorithm used for encryption is the Salsa20 stream cipher by Daniel J. Bernstein \cite{bernstein2008salsa}.
It is a family of 256-bit stream ciphers to designed to be used in a wide range of cryptographic applications.
It uses a 256-bit key and a 64-bit nonce and expands them into a $2^{70}$-byte stream.
Encryption for a plain text message with $n$ bytes is done by xor'ing the plaintext with the first $n$ bytes of the stream.
Decryption is done likewise, xor'ing the ciphertext with the first $n$ bytes of the same stream.
We specifically use the Salsa20/20 stream cipher, which is a 20-round stream cipher, compared to reduced-round ciphers specified by Daniel J. Bernstein.

The algorithm used for message authentication is the Poly1305 message authentication code by Daniel J. Bernstein \cite{bernstein2005poly1305}.
Poly1305 takes a 32-byte key and a message and produces a 16-byte tag that authenticates the message.
The key for the combination of Salsa20/Poly1305 is generated by deriving a subkey from a tuple of key and nonce used for encrypting the actual message.

\subsubsection{Message exchange}

Exchanging data between hosts poses several questions.
Consider we want to send a data structure over to another host.
The naive approach would be to simply stream over the in-memory representation of the structure and let the receiver interpret the data stream as the corresponding structure.
This naive approach is not even guaranteed to work for two hosts running on the same architecture, as for example the compiler may choose to re-arrange structure entries.
The approach finally falls flat when considering different operating systems with different application binary interfaces or even different memory layouts, e.g. different byte orders.

The problem of correctly serializing data is not trivial and thus we choose to not handle the problem ourselves but let a library designed for this problem handle serialization.
Google Protocol Buffers \cite{varda2008protocol} implement a method for serializing structured data in platform- and language-independent way.
Protocol Buffers come with a way to specify interface descriptions that describe how data is structured in a certain message type.
These interface descriptions are then compiled to language-specific code that can be imported via language-specific constructs.

One inherent benefit of this approach is that we are able to re-use the interface definitions across multiple sites.
In particular, we use the same interface definitions across the command line client and server which are both written in C, and the Android controller which is written in Java.

The whole protocol is built upon these interface descriptions with the only exception being when we forward application-specific data between two specific services.
As we do not want to know about the specific protocols they use in order to be generic as possible we simply relay data from and to both service end points.
Otherwise all message exchanges boil down to serializing data with Protocol Buffers and then de-serializing them on the receiving site.

\subsection{Protocol}

We will now discuss the application level protocol as it is used by servers and clients to communicate with each other.
Everything discussed in here will use the low-level protocol discussed in section \ref{sec:low-level-protocol}.

\subsubsection{Connection establishment}
\label{sec:connection-establishment}

When a client connects to a service we want to encrypt encrypt their communication.
Obviously we need to have a way to negotiate the encryption between both parties, as the server has to know who is about to connect and the client wants to authenticate that he is really talking to the server he wants to talk to.
We will now consider the key exchange protocol used to negotiate ephemeral encryption key used for the connection.

The key exchange assumes that the public identities, that is the public signature keys of both parties, are known to each other.
They are used to verify the actual identity of the opposite party.

The sequence diagram in figure \ref{fig:connection-establishment} provides an overview of the key negotiation between two parties, the client $C$ and the service $S$, where $C$ has the intent to connect to $S$.
Let us first define all functions used in the diagram.

\begin{description}
    \item[Sign$_s(m)$]\hfill\\
        The signing algorithm generates upon input of a private signature key $s$ and message $m$ a signature $\sigma \leftarrow \text{Sign}_{s}(m)$.
    \item[Verify$_v(m, \sigma)$]\hfill\\
        The verification algorithm takes as input a public verification key $v$, a message $m$ and a signature $\sigma$.
        Verify returns success iff the signature $\sigma$ is valid for the verification key $v$ and message $m$.
    \item[GenerateKeypair()]\hfill\\
        The key generation algorithm outputs a pair of keys $(enc, dec)$, where $enc$ is the public encryption key and $dec$ is the private decryption key.
        Keys are generated with a cryptographically secure random number generator.
    \item[CalculateKey($enc, dec$)]
        The key calculation algorithm generates upon input of a local private encryption key and a remote public decryption key a new symmetric key via Diffie-Hellman key generation.
\end{description}

First the client $C$ generates a new ephemeral key pair $(enc_C, dec_C) = \text{GenerateKeypair}()$.
He then proceeds to send an initial message containing his public verification key $vrfy_C$, the newly generated public ephemeral key $enc_C$ and a signature of $vrfy_C$ and $enc_C$ concatenated.

Upon receiving the initial message, the service verifies that the concatenated keys have indeed been signed by the announced client's public verification key.
This step verifies that the received message has at one point been in fact generated and signed by an entity which possesses the key $vrfy_C$.
If the verification is successfull, we continue and generate our own ephemeral keypair $(enc_S, dec_S) = \text{GenerateKeypair}()$.

The service now sends a signature of the generated ephemeral keys of both client and service as well as both keys in plain text.
Back on the client side we can now verify that we are indeed talking to the correct identity, as the signature now contains not only the remote's newly generated ephemeral public key, but also our own randomly generated public key.
Due to this it is safe to assume that the service indeed posseses the private signature key belonging to the public signature key we know.

At this point of time we have verified that we are actually talking to an entity possessing $vrfy_S$, but the server does not yet know that he is indeet talking to a client possessing $vrfy_C$, as the initial message could have been recorded and replayed by an adversary.
To fix the situation we now perform the same step of signing both public ephemeral keys and sending them to the server again, who may now verify whom he is talking to.

When both parties are verified to each other we can now calculate the symmetric key which is used for subsequent communication.
For each party, the key is calculated by performing a scalar multiplication between the party's private ephemeral key and the remote's public ephemeral key.
The output is then concatenated with both parties' public keys' and hashed, resulting in a new secret which is shared between both client and server.

\begin{figure}
    \centering

    \begin{sequencediagram}
        \newthread{c}{Client ($vrfy_C$, $sign_C$)}
        \newthread[4]{s}{Service ($vrfy_S$, $sign_S$)}

        \begin{call}{c}{GenerateKeypair()}{c}{($enc_C$, $dec_C$)}
            \postlevel
        \end{call}

        \postlevel

        \begin{messcall}{c}{$\sigma = \text{Sign}_{sign_C}(vrfy_C \| enc_C), vrfy_C, enc_C$}{s}
        \end{messcall}

        \begin{call}{s}{Verify$_{vrfy_C}(\sigma, vrfy_C \| enc_C)$}{s}{}
            \postlevel
        \end{call}

        \postlevel

        \begin{call}{s}{GenerateKeypair()}{s}{($enc_S$, $dec_S$)}
            \postlevel
        \end{call}

        \postlevel

        \begin{messcall}{s}{$\sigma = \text{Sign}_{sign_S}(enc_S \| enc_C), enc_S, enc_C$}{c}
        \end{messcall}

        \begin{call}{c}{$\text{Verify}_{vrfy_S}(\sigma, enc_S \| enc_C)$}{c}{}
            \postlevel
        \end{call}

        \postlevel

        \begin{messcall}{c}{$\sigma = \text{Sign}_{sign_C}(enc_S || enc_C), enc_S, enc_C$}{s}
        \end{messcall}

        \begin{call}{s}{$\text{Verify}_{vrfy_C}(\sigma, enc_S \| enc_C)$}{s}{}
            \postlevel
        \end{call}

        \postlevel

        \begin{call}{c}{CalculateKey($enc_S$, $dec_C$)}{c}{Key}
            \postlevel
        \end{call}

        \prelevel
        \prelevel
        \prelevel

        \begin{call}{s}{CalculateKey($enc_C$, $dec_S$)}{s}{Key}
            \postlevel
        \end{call}
    \end{sequencediagram}

    \caption{Connection establishment}
    \label{fig:connection-establishment}
\end{figure}

\subsubsection{Discovery}

Device discovery is the process of discovering previously unknown services in a certain subnet.
That is given a client and a certain amount of services that all reside in the subnet we want to get to know all the services that are present without previously knowing where they are.

Let us consider the two obvious solutions coming to mind, that is client- and service-driven approaches.
\begin{itemize}
    \item The service-driven approach is based on the service periodically broadcasting its availability to the subnet, thereby notifying possibly listening clients of the availability of a certain service.
    \item The client-driven approach on the other side lets the client query the subnet for existing services by announcing that it wants to know about available services.
        Every service listening to these announcements will receive the query and subsequently announce its availability to the server.
\end{itemize}

Both methods have their own merit.
Periodic announcements of the service conceal the intent of clients to discover local services as they are not directly involved in the process of finding services.
They have the disadvantage of additional overhead, though, as we regularly have to announce our service to the whole subnet.
While this might not be problematic when only few services are available in a certain subnet it may become unwieldy when hundreds of services are present which are constantly firing announce messages to each member of the network.

On the other hand with a client-driven approach it is easy to avoid this particular problem.
Given a client which wants to probe the network for available services we will simply drop a broadcast message to the subnet and wait for available services to respond.
Like this the whole process is intent-driven, as the user has an explicit intent of discovering services, and repeating messages are avoided.

Just how important this distinction is gets particularly clear when we consider the process how users want to start interacting with a certain service.
Assume Alice starts up her mobile phone and intents to discover a certain service she expects to be available in the local network.
Given the approach where services periodically announce their availability, Alice might have to wait the whole time span between two announcements if she just missed the previous one.
Depending on how the announcement interval was chosen this may be a long time.
The worst case scenario is when Alice becomes too uncomfortable waiting for the first availability announment to arrive, instead simply dropping her intent and aborting discovery.

When we actively probe the subnet for available services instead the response times will in the mean case be much less than with the other approach.
That is the delay of services popping up is only based on latency and processing time of both service and client.
As a consequence we have chosen to implement the client-driven approach of actively probing the subnet.\\

%%%% TODO: proof-read from here on

The initial exchange between a client and service which have never seen each other before is somewhat problematic, as in general we cannot establish the authenticity of either one.
To accomodate for this we distinguish between undirected (see figure \ref{fig:undirected-service-discovery}) and directed service discovery(see figure \ref{fig:directed-service-discovery}).
While the undirected discovery protocol is unauthenticated we are able to improve the situation for directed service discovery in certain cases.

Through a sidechannel we are able to retrieve additional data about a service we want to connect to.
As described by McCune et al. \cite{mccune2005seeing} we can establish additional trust amongst other things by scanning a QR code that is either printed on the case of the service or, further improving the trust level, by dynamically displaying a QR code on the screen.
This is particularly interesting in the case where we want to connect to a display service.
Upon notifying the service in a certain out-of-band way (e.g. by pressing a certain button on the display) it will generate a QR code and display it on the screen.
The client can subsequently scan the code and in such a way retrieve information like the public key and address of the display.
As soon as we have obtained data through this side channel we are able to verify authenticity of the service we are communicating with, assuming the side channel has not been compromised.

Let us now consider the actual discovery protocol.

\begin{figure}[H]
    \centering

    \begin{sequencediagram}
        \newinst{s1}{Service 1}
        \newthread[4]{c}{Client}
        \newinst[4]{s2}{Service 2}

        \mess{c}{Discover}{s1}
        \prelevel
        \mess{c}{Discover}{s2}

        \postlevel

        \mess{s1}{Announce}{c}
        \mess{s2}{Announce}{c}
    \end{sequencediagram}

    \caption{Service Discovery}
    \label{fig:undirected-service-discovery}
\end{figure}

We assume both client and service own a long term key pair $(pk_c, sk_c)$ and $(pk_s, sk_s)$, respectively, that is used for establishing their own identity.
The initial \emph{Discover} message contains a protocol version, port and the public key of the client.

The version is required as future iterations on the protocol may lead to incompatibilities between different versions of the software stack that is in use on both client and service.
It is of utmost importance to check those versions for both parties to avoid running into incompatibilities that may cause the protocol to seemingly work while in fact it is broken in a subtle way.

When the service receives a UDP message the originating address is recorded in this message.
When we want to respond to that message, though, we are not able to do so without also knowing which port to send that response to, as UDP is a connection-less protocol.
As such we include a listing port inside the initial message that will be used by the service to send the response message.

Last we also include the client's long term public key to establish the client's identity and which will be used in future exchanges to encrypt further messages.
The whole message is then signed by the client and sent to the service.\\

On the receiving side we first unpack the message to retrieve version, port and public key.
With the public key and the message's tag we are now able to verify the integrity of the received package.
Obviously, though, we are not able to say anything about the authenticity about it as it could have been intercepted by a malicious third party, exchanging the message tag and public key.

After establishing message integrity the service will now compare the client's version to its own version and only proceed if he is able to correctly service that version.
He will then assemble a response package containing its own version, name, available local services and its public key.

It is possible for a single server to announce multiple services.
Those services may be distinguished by the port they are listenging at.
Instead of announcing multiple service types at a single port, which would then require multiplexing requests to their respective service and thus introduce additional complexity, each service has its own port which will later be used to connect to that service.

% TODO: encrypt message with public key?
The server will now again generate a message tag and send both message and tag encrypted with the public key of the client to the client's listening port that has been previously announced.

In contrast to the indirected service discovery, there is only one significant difference in the directed service discovery.
As we have previously obtained public key data of the server we want to talk to we are already able to use its key to encrypt the first message.
That is instead of sending $m \| t$ in the fisrt step, we send $\text{Enc}_{pk_s}(m \| t)$.
Like this it is impossible to tamper with the first message in such a way that the receiver retrieves a valid message with invalid key data.

All subsequent steps remain unchanged.

\subsubsection{Query}

\begin{figure}[H]
    \centering

    \begin{sequencediagram}
        \newthread{c}{Client}
        \newinst[8]{s}{Server}

        \begin{messcall}{c}{InitiateConnection()}{s}
            \postlevel
            \begin{call}{c}{Query()}{s}{Service description}
            \end{call}
        \end{messcall}

    \end{sequencediagram}
    \caption{Query Diagram}
\end{figure}

\subsubsection{Session Initiation}

\begin{align*}
    \text{capability} \coloneqq pk_s(\text{sessionid}, \text{parameters})
\end{align*}

\begin{figure}[H]
    \centering

    \begin{sequencediagram}
        \newthread{c}{Client}
        \newinst[8]{s}{Server}

        \begin{messcall}{c}{InitiateConnection()}{s}
            \postlevel
            \begin{call}{c}{InitiateSession(parameters)}{s}{Session Capability}
            \end{call}
        \end{messcall}

        \postlevel

        \begin{messcall}{c}{InitiateConnection()}{s}
            \postlevel
            \begin{messcall}{c}{Connect(session capability)}{s}
            \end{messcall}
        \end{messcall}

        \prelevel
    \end{sequencediagram}
    \caption{Direct Connect}
\end{figure}

\subsection{Services}

\subsubsection{Capability Service}

\begin{figure}[H]
    \centering

    \begin{sequencediagram}
        \newthread{r}{Requester r}
        \newinst[2]{e}{Entity e}
        \newinst[2]{s}{Service s}
        \newinst[2]{c}{Capability Server}

        \mess{e}{Register}{c}
        \postlevel

        \begin{call}{r}{Request(e, s, params)}{c}{Service Session}
            \postlevel
            \begin{call}{c}{Ask(r, s, params)}{e}{Service Session}
                \postlevel
                \begin{call}{e}{Initiate(params)}{s}{Service Session}
                \end{call}
                \postlevel
            \end{call}
            \postlevel
        \end{call}

        \postlevel

        \begin{messcall}{r}{Start}{s}
            \postlevel
        \end{messcall}

        \prelevel
    \end{sequencediagram}
    \caption{Capability Request}
\end{figure}

\subsubsection{Invoke Service}

\begin{figure}[H]
    \centering

    \begin{sequencediagram}
        \newthread{c}{Client}
        \newinst[4]{s}{Service}
        \newinst[4]{i}{Invoker}

        \begin{call}{c}{Initiate(Parameters)}{s}{Service Session}
            \begin{call}{s}{CreateSession()}{s}{Session}
            \end{call}
        \end{call}

        \postlevel

        \begin{call}{c}{Initiate(ServiceSession)}{i}{Invoker Session}
            \begin{call}{i}{CreateSession()}{i}{Session}
            \end{call}
        \end{call}
        \postlevel

        \begin{messcall}{c}{Start}{i}
            \begin{messcall}{i}{Start}{s}
                \postlevel
            \end{messcall}
            \prelevel
        \end{messcall}
        \prelevel
    \end{sequencediagram}

    \caption{Invoke Service}
\end{figure}

\subsubsection{xpra Service}

\begin{figure}[H]
    \centering

    \begin{sequencediagram}
        \newinst{x}{Xpra server}
        \newthread[4]{c}{Client}
        \newinst[4]{s}{Display service}

        \begin{call}{c}{Start(port)}{x}{instance}
        \end{call}

        \postlevel

        \begin{messcall}{c}{InitiateConnection()}{s}
            \postlevel
            \begin{call}{c}{Request(xpra-port)}{s}{session}
            \end{call}
        \end{messcall}

        \postlevel

        \begin{messcall}{c}{InitiateConnection()}{s}
            \postlevel
            \begin{messcall}{c}{Connect(session)}{s}
            \end{messcall}

            \postlevel

            \begin{messcall}{s}{Tunneled data exchange}{x}
                \postlevel
            \end{messcall}
            \prelevel
            \prelevel
            \prelevel
            \begin{messcall}{x}{}{s}
            \end{messcall}
        \end{messcall}
    \end{sequencediagram}

    \caption{Xpra Service}
\end{figure}

\subsection{Notes}

\begin{itemize}
    \item host has graphic server
    \item only knows about window extents, not contents due to encryption
    \item initially gets all keyboard/mouse inputs
    \item upon mouse enter inside a window, notifies mouse server of window extents
    \item mouse server keeps track of relative mouse position, notifies server upon exit of extents
    \item window server registers shortcuts that are required for window changing
    \item shortcuts may NEVER be rebound by client applications

    \item you can attach and reattach apps to other devices without any problems!!!
    \item overlay in server manager when app is connected and reqs are not fulfilled
    \item one container may run multiple apps
        \begin{itemize}
            \item clustering based on security level
        \end{itemize}
\end{itemize}

% vim: ft=tex tw=0
